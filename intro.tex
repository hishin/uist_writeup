\section{Introduction}
Audio recordings are a common form of communication used in voice-overs, podcasts, audio books and e-learning. Closest to everyday conversation, audio recording\ is a medium with relatively low barriers to entry. So, it is used by many laymen who are not professional producers or writers to distribute information. A common workflow for creating such audio recordings involves three main steps: writing a script, recording audio and editing audio. To create a compelling audio recording, users go through several iterations of these steps.  

Consider the case of recording a voice-over for a video. The narrator does an initial recording based on a prepared script. Afterwards, while placing it on top of the video, the producer wants to make some changes to parts of the narration e.g. to control the timing of a specific word, or to match changes made in the shots after the audio was recorded. While some of these edits can be done from the existing recording using an audio editing software, others require editing the script and re-recording the altered parts. Similarly, consider an audio recording of an online lecture. After the initial publication, the lecturer may want to re-record or add parts e.g. to keep examples up to date, or to address common questions that came up from viewers. These are but a few of many scenarios where producers iterate back and forth between script writing, audio recording and audio editing.

Most existing audio editing systems provide functionalities necessary
to support the latter two steps: recording and editing audio.
However, the first step, writing or editing the script, is usually overlooked
or treated as a completely separate task. This is the case even
when scripts play a key role in recording and editing speech.
  
% Moreover, traditional tools provide many features that are useful for manipulating audio waveforms, but that are not directly relevant for creating audio stories. Their complex user interface can be a overwhelming for novice users. 

In this paper, we present an interface that supports and links all three steps of the aforementioned workflow. Our system addresses challenges that span the process of creating audio narratives, including (1) iterating back and forth between script  and audio recordings, and (2) combining multiple audio recordings into a single final track. Our interface is inspired from familiar document editors and text merge tools, which are easy to learn.

