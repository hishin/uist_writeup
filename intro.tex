\section{Introduction}
Audio recordings are a common form of communication used in voice-overs, podcasts, audio books and e-lectures. Closest to everyday conversation, audio recording\ is a medium with relatively low barriers to entry. It is used by many laymen who are not professional producers or writers. A common workflow for creating such audio recordings involves three main tasks: writing a script, recording audio and editing audio. In many cases, users go back and forth between these tasks in order to create the final audio. 

Consider the case of recording the audio for an online lecture. The lecturer prepares lecture notes or slides and uses it as a rough script while recording. After recording a couple of takes, she decides that it would be useful to insert a further explanation about one of the points. She edits the notes and re-records that part. The final audio is created by cutting and merging the multiple recordings. After the initial release on an online platform, viewers leave feedback. The lecturer realizes that many people are confused about a particular point. In order to address their concern, she revises the notes and also  re-records that part with a new explanation and more examples. The audio is re-edited and updated. 

Similarly, consider recording a voice-over for a video. The narrator does an initial recording based on a loosely prepared script. Afterwards, while placing it on top of the video , the narrator realizes that additional shots are needed to make the narrative clear. New shots are inserted, the script is edited to include matching narrative and parts of the audio is re-recorded.
These are but a few of many scenarios where
users go back and forth between script writing, audio
recording and audio editing.



Most existing tool for script writing and audio recording/editing treat the two resources (i.e. script and audio) as completely separate. Users create and edit the script document using one tool, and record and edit the audio using another tool.
The task of making a connection between the script and the audio is left for the user to do manually. This is the case even
when for audio where scripts play a key role.
  
% Moreover, traditional tools provide many features that are useful for manipulating audio waveforms, but that are not directly relevant for creating audio stories. Their complex user interface can be a overwhelming for novice users. 

We present an interface that links and supports all of the three main tasks in audio production: script writing, audio recording and audio editing. Our system addresses challenges that span the process of creating audio recordings, including (1) linking the script  with audio recordings, (2) supporting a dynamic workflow, and (3) merging multiple audio segments into a single final track. Our interface is inspired from familiar document editors and text merge tools, which are easy to learn.

