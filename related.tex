\section{Related Work}

Adobe Story \cite{adobestory2016}, FinalDraft \cite{finaldraft2016} and Celtx \cite{celtx2016} are examples of software applications dedicated to script writing. They support collaboration, automatic formatting, navigation and planning for future production, but they treat the script as a text document that is essentially separate from the recordings. In fact, in our preliminary interview of lay and professional audio producers, we found that many of them use general-purpose document editors like Google Docs \cite{googledocs2016} or Microsoft Word \cite{microsoftword2016} to prepare their scripts.

At the recording and editing stage, Adobe Audition \cite{adobeaudition2016}, Avid ProTools \cite{avidprotools}, GarageBand \cite{garageband} and Audacity \cite{audacity} are among the most popular digital audio workstations (DAWs). These tools allow users to edit audio by manipulating waveforms in a multi-track timeline interface. They also provide a wide variety of low-level signal processing functions. However, since they are designed to serve as general-purpose audio production systems, they include many features that are not directly relevant for creating audio narratives whose main content is speech. Hindenburg Systems \cite{hindenburg} develops tools that are specifically targeted for audio narratives. Still, they are primarily concerned only with the audio and they do not deal with the script directly.   

Recently, several researchers have explored using audio transcripts to support text-based navigation and editing of audio. Whittaker and Amento \cite{whittaker2004semantic} demonstrate that users prefer editing voicemail through its transcript instead of its waveform. Inspired by similar intuition, Casares et al. \cite{casares2002simplifying} and Berthouzoz et al. \cite{berthouzoz2012tools} enable video navigation and editing through time-aligned transcripts. Rubin et al. \cite{rubin2013content} extend this approach to audio narratives and propagate edits in the transcript text to the corresponding speech track. These systems all focus on editing the audio on the assumption that all the recording is done beforehand. \textit{Narration Coach} developed by Rubin et al. 
supports an iterative narration recording process, but instead it assumes that the user has a fixed input script. Its focus, providing capture-time feedback to improve speech performance, is also very different from ours. Our work also takes advantage of text-based navigation and editing, but unlike these systems, we support an dynamic workflow where both the audio recordings and the underlying script can be continuously  updated.      