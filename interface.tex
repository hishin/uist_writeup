\section{User Interface}
Motivated by these insights, we developed the \systemname\  interface to support iterative script writing and  audio recording/editing. As shown in Figure~\ref{interface}, our interface consists of two types of documents: a \textit{Master Script} document (left) displays the current status of the final track, including what has been recorded and what was planned to be recorded but has not been recorded yet (i.e. the original script), while a \textit{Transcript} document (right) displays the verbatim transcript of individual audio takes. As the user records new takes, our tool aligns the audio transcripts to the master script so that it is easy to compare each take with the master script and with each other. It also partitions each recording into segments that can be seamlessly joined between takes. At any point during the recording process, the user can edit the script or the final track by editing the master script like a text document. The rest of the section describes the interface using an example workflow. 

In Figure~\ref{masterscript}, the user begins by writing on the master script an outline  of points to record. The text appears in light grey to indicate that these parts have not been recorded yet. At this stage, the master script is like an ordinary
word document or script. 

Once the user starts recording, the audio is transcribed in real time and a verbatim text corresponding to each take appears in a separate transcript document tab. Each transcript is time-aligned with the corresponding recording, so that users can quickly navigate to specific
part of the audio by clicking on a word in the transcript.

If the \textit{compare-view} is turned on, our system aligns segments of the transcript to corresponding segments in the master script, in this case the points in the original outline.

% partitions the transcript and the master script into short segments, and aligns them to each other. The boundaries of these segments are computed by taking into account, along with other factors,  pauses in the audio, such that when different audio segments corresponding to the transcript segments are joined together the cut is seamless.

When the user has multiple takes, in addition to each of the transcripts, the \textit{all-tab} provides a summary of all of the takes (Figure~\ref{all-tab}). As in the \textit{compare-view}, for each segment in the master script, it displays the corresponding transcript segments, this time from all of the audio takes. A drop-down button indicates that there are multiple versions (or takes of the same segment. Clicking on the button opens a
list showing the alternatives versions from different takes (Figure ~\ref{alternate-view}). The user can listen to any of these takes and select one without having to search through individual takes. 

Now that part appears in text color to indicate. If the user has multiple takes, they can go back and forth. If they go to the All Tab, they can see multiple takes of each sentence.
Also edit the master script, just as they would edit a text document. If the user edits a recorded portion of the script, the relevant section is makred as dirty to indicate that it should be re-recorded.

