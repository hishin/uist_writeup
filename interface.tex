\section{User Interface}
Motivated by these insights, we developed the \systemname\  interface to support iterative script writing and  audio recording/editing. As shown in Figure~\ref{interface}, our interface consists of two types of documents: a \textit{Master Script} document (left) displays the current status of the final track, including what has been recorded and what was planned to be recorded but has not been recorded yet (i.e. the original script), while a \textit{Transcript} document (right) displays the verbatim transcript of individual audio takes. As the user records new takes, our tool aligns the audio transcripts to the master script so that it is easy to compare each take with the master script and with each other. It also partitions each recording into segments that can be seamlessly joined between takes. At any point during the recording process, the user can edit the script or the final track by editing the master script like a text document. The rest of the section describes the interface using an example workflow. 

In Figure~\ref{masterscript}, the user begins by writing on the master script an outline  of points to record. The text appears in light grey to indicate that these parts have not been recorded yet. At this stage, the master script is like an ordinary
word document or script. 

Once the user starts recording, the audio is transcribed in real time and a verbatim text corresponding to each take appears in a separate transcript document tab. Each transcript is time-aligned with the corresponding recording, so that users can quickly navigate to specific
part of the audio by clicking on a word in the transcript. If the \textit{compare-view} is turned on, our system aligns segments of the transcript to corresponding segments in the master script, (in this case the points in the original outline), and shows them side-by-side (Figure~\ref{compare-view}). If a segment in the transcript does not correspond to any of the segments in the master-script (e.g. when the speaker improvises or ad-libs), it is highlighted in yellow.

% partitions the transcript and the master script into short segments, and aligns them to each other. The boundaries of these segments are computed by taking into account, along with other factors,  pauses in the audio, such that when different audio segments corresponding to the transcript segments are joined together the cut is seamless.

When the user has multiple takes, in addition to each of the transcripts, the \textit{all-tab} provides a summary of all of the takes. As in the \textit{compare-view}, for each segment in the master script, it displays corresponding transcript segments, this time from all of the audio takes. A drop-down button indicates that there are multiple versions (or takes) of the same segment. Clicking on the button opens a
list showing the alternative versions from different takes (Figure ~\ref{alternate-view}). The user can listen to any of these takes and select one without having to search through individual takes. If none of the recordings contains a part corresponding to a segment in the master script, it is highlighted in red. In this way, the user can easily keep track of what has been recorded, and what still needs to be recorded.  

The user can \textit{accept} audio segments into the final track by clicking on a button next to each transcript segment either from the individual transcript tab or the all-tab. If there is a corresponding master script segment, the transcript segment replaces it. If the segment was improvised the transcript segment is simply inserted into the master script.
The text of this segment in the master script now appears in darker color to indicate that it has been recorded. Now, the master script contains both recorded text that is accepted to the final track, and unrecorded text from the original script.
Playing the master script will play only the darker, recorded text. 

 At any stage, the user can edit the master script like a text document. Edits made to the recorded part of the text is reflected in the audio. For instance, the user can delete a recorded word from the text, and it will be deleted from the audio. \VTODO{Mark as dirty. Correct transcription.} Users can also simply insert more text to record (which appears in light grey), or make changes to unrecorded text, for example, to flesh out the original outline or change the wording of a particular sentence. 

The user iterates back and forth between all of these steps, editing the master script, recording audio takes, and accepting segments of the takes into the final track to produce the final recording.
The beauty of our interface is that it supports a wide range of workflow that each user can adapt to different scenarios. For instance, instead of starting with a written outline, the user can begin with an empty script and simply start recording,  then use  the initial recording as an outline. The user can also record the entire script in a single take, or work on a single section at a time and then assemble them together to construct a narrative.
