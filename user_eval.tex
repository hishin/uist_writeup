\section{Informal User Evaluation}
\VTODO{Table of usage statistics from both studies.}

To gauge the utility of our interface, we conducted an informal evaluation with two users (U1 and U2). We started each session with a 10-minute demonstration of interface. Then, we gave them a short article about a technical subject and asked them to create an explanatory audio recording using our authoring interface. They were allowed to go back to the article during the authoring process or to take notes on the master-script, but they were discouraged from recording the article by reading it out loud. We examined their workflow, and the number/type of features they used. We also solicited written qualitative feedback about our interface at the end of the session. Each session lasted about 40 minutes.

U1 created a recording about the article \textit{What is a decibel? } from howstuffworks.com \cite{}, and U2 created a recording about \textit{How lasers work} from David Macaulay's illustrated book, \textit{The Way Things Work} \cite{}. Overall, the results from the study were extremely encouraging. Both users successfully produced a complete audio recording summarizing the articles, taking advantage of many of our interface features (Table~\ref{}). 

Interestingly, each user adapted a very different workflow. U1 started by writing a complete outline as a list of main ideas. For each take, U1 recorded a few points in the outline, merged them into the master-script then continued to record the next point on a separate take. After merging in all of the points in the final track, U1 noticed a mistake in one of the examples (instead of saying \textit{140 decibels}, U1 had said \textit{40} decibels). U1 corrected the corresponding recorded text in the master-script which was consequently marked red. U1 re-recorded and replaced this part by reading out the edited master-script. On the other hand, U2 wrote part of the outline, recorded that portion, and moved onto write the outline of the next part. The different workflow could be due to personal preference, or  to the fact that U2's article was organized into a clear step-by-step explanation whereas U1's article was flowed like a continuous narrative. In either case, our interface was able support both users' workflow. 

?

Both users also offered strong positive feedback about our authoring
interface. They were most enthusiastic about the integration
of the script and the final track in the master-script view,
and the ability to align the master-script to the transcripts.
U2 wrote, \textit{``Writing the outline on the UI and having
that integrated with the audio was most helpful.''}  U1 said
that the alignment, \textit{``helped to keep track of what pieces
of information was already recorded and which ones were still
needed.''} Participants were also impressed with the quality
of the merged recording. U1 wrote, \textit{``I was surprised
how the final recording from the multiple takes was seamless.''}  
\VTODO{Explain U1 workflow, U2 workflow}

\VTODO{Qualitative feedback}


We also conducted a pilot study to compare our interface with a state-of-the-art transcript-based speech editing interface \cite{rubin2013content}. \VTODO{Briefly explain Steve Rubin's interface.} We recruited four participants, none of whom had experience using text-based audio editing systems. We gave them a script with bullet points outlining a mini lecture on a science subject (e.g. \textit{gravity} and \textit{dark matter}) and two audio takes roughly corresponding to that script. Their task was to cut and merge the two takes to produce a recording that contained all the contents listed in the script and only those contents. The two takes were similar, but both takes had some
missing content from the outline and one had some extra
content. So, the participants had to choose parts from each take and combine them to get the final result. We encouraged the users to focus on having the complete content, rather than the details of the audio quality (e.g. tempo, diction, flow of speech etc.). 

Each participant completed the task twice with different outlines, once using our interface and the other using Rubin et al.'s interface. The subject of the outline and the order of the interface was counter-balanced. We examined the time time they spent editing, the number/type of functions they used, and the quality of the final recording. After the session, participants gave written qualitative feedback about the two interfaces. In total, each session lasted one hour.
   
Each of the four participants preferred \systemname\ over Rubin et al.'s interface for the given task, and noted they would use our interface to edit audio recordings. Every participant also completed the task faster using our interface (avg. 7.4 $\pm$ 1.6 min) than Rubin et al's interface(9.9 $\pm$ 1.5 min). Table~\ref{evalsumary2} summarizes the participants' usage of \systemname\ during the editing session.

\VTODO{Explain usage}

\VTODO{Positive Feedback}
