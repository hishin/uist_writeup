\section{Informal User Evaluation}
To gauge the utility of our interface, we conducted an informal evaluation with two users. We started each session with a brief demonstration of our interface, and then asked the participants to create an audio recording. We examined their workflow, and the number/type of features they used. We also solicited written qualitative feedback about our interface at the end of the session. Each session lasted about 50 minutes.


We also conducted a pilot study to compare our interface with a state-of-the-art transcript-based speech editing interface \cite{rubin2013content}. We recruited four participants, none of whom had experience using text-based audio editing systems. We gave them a script with bullet points outlining a mini lecture on a science subject (e.g. \textit{gravity} and \textit{dark matter}) and two audio takes roughly corresponding to that script. Their task was to cut and merge the two takes to produce a recording that contained all the contents listed in the script and only those contents. The two takes were similar, but both takes had some
missing content from the outline and one had some extra
content. So, the participants had to choose parts from each take and combine them to get the final result. Each participant completed the task twice with different outlines, once using our interface and the other using Rubin et al.'s interface. The subject of the outline and the order of the interface was counter-balanced. After the session, participants gave written qualitative feedback about the two interfaces. In total, each session lasted one hour.   
 
