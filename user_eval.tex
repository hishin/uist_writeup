\section{Informal User Evaluation}

\begin{table*}[ht]
\center
\tabcolsep10pt
\tbl{Audio Recording Session}{\\%
\begin{tabular}{c|cccccccc}\hline
{Subject}&{Takes} &{Total}&{Edited} & {Total} & {Accept}
&{View}&{View}&{Text} \\
{}&{recorded}&{length}&{length}&{cuts}&{segment}&{re-takes}&{diff}&{edit} \\\hline
\textit{What is a decibel?}&{5}&{1:37}&{0:51}& {5}&{5} &{-}&{-}&{3}\\
\textit{How lasers work}  &{4}&{2:49}&{1:06}&{6}&{22}&{5}&{-}&{15}
\\\hline
\end{tabular} }
\label{Tab:TiltTest}
\end{table*}
To gauge the utility of our interface, we conducted an informal evaluation with two users (U1 and U2). We started each session with a 10-minute demonstration of our interface. We gave them a short article about a technical subject and asked them to create an explanatory audio recording using our authoring interface. Users were allowed to go back to the article during the authoring process or to take notes on the master-script, but they were discouraged from recording the article by reading it out loud. We examined their workflow, and the number/type of features they used. We also solicited written qualitative feedback about the authoring experience at the end of the session. Each session lasted about 40 minutes.

U1 created a recording about the article \textit{What is a decibel? } from howstuffworks.com \cite{howstuffworks}, and U2 created a recording about \textit{How lasers work} from David Macaulay's illustrated book, \textit{The Way Things Work} \cite{macaulay1999way}. Overall, the results from the study were extremely encouraging. Both users successfully produced a complete audio recording summarizing the articles, taking advantage of many of our interface features (Table~\ref{}). 



Interestingly, each user adapted a very different workflow. U1 started by writing a complete outline as a list of main ideas. For each take, U1 recorded a few points in the outline, merged them into the master-script then continued to record the next point on a separate take. On the other hand, U2 wrote part of the outline, recorded that portion, and moved on to write the outline of the next part. The different workflows could be due to personal preference, or  to the fact that U2's article was organized into a clear step-by-step explanation whereas U1's article flowed like a continuous narrative. In either case, our interface was able support both users' workflow. 

Both users took advantage of the master-script to go back and forth between scripting and audio recording. After recording and merging in all the points
to the final track, U1 noticed a mistake in one of the examples
(instead of saying \textit{140 decibels}, U1 had said \textit{40}
decibels). U1 corrected the corresponding recorded text in the
master-script, which was consequently marked red. U1 re-recorded
and replaced this part by reading out the edited master-script.
U2 wrote a very rough outline for the first part of the article. Then U2 used the transcript of the first take to refine the script, and recorded a second take. For the second part of the article, U2 wrote a much more detailed script beforehand. U2 noted,  \textit{``For the introduction, I had a pretty good idea of what I wanted to say, so it saved me time to use only bullet points. [For the second part] I wrote full sentences, as I was not familiar with all the technical details and it would have been more difficult
to improvise. I enjoyed being able to use the master-script in both ways.''
}


Both users offered strong positive feedback about our authoring
interface, and said they would use it for creating speech recordings. They were most enthusiastic about the integration
of the script and the final track in the master-script view,
and the ability to align the master-script to the transcripts.
U2 wrote, \textit{``Writing the outline on the same interface and having
that integrated with the audio was most helpful.''}  U1 said
that the alignment \textit{``helped to keep track of what pieces
of information was already recorded and which ones were still
needed.''} Participants were also impressed with the quality
of the merged recording. U1 wrote, \textit{``I was surprised
how the final recording from the multiple takes was seamless.''} U2 noted that while speech recognition was imperfect, \textit{the transcriptions were accurate enough to understand and easy to check [with the audio].''}

