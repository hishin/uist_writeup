\section{Informal User Evaluation}

% \begin{table*}[ht]
% \center
% \tabcolsep5pt
% \begin{tabular}{c|cccccccc}
% \multicolumn{9}{c}{\textbf{Audio Authoring Session}}\\\hline
% {Subject}&{Takes} &{Total}&{Edited} & {Total} & {Accept}
% &{View}&{View}&{Text} \\
% {}&{recorded}&{length}&{length}&{cuts}&{segment}&{alternate}&{diff}&{edit} \\\hline
% \textit{What is a decibel? (U1)}&{5}&{1:37}&{0:51}& {5}&{5} &{-}&{-}&{3}\\
% \textit{How lasers work (U2)}  &{4}&{2:49}&{1:06}&{6}&{22}&{5}&{-}&{15}
% \\\hline
% \end{tabular} 
% \label{tab:recording}
% \caption{Two users (\textit{U1} and \textit{U2}) created audio recordings using our system and we collected their usage statistics. Total cuts refer to the number of places where audio segments from different takes were joined together. View alternate refers to the number of times users viewed alterantive takes of a segment in  \textit{all-tab} using a drop-down button.}
%\end{table*}

To assess the overall usability of \systemname\ and to observe how users leverage various features of the interface, we conducted an informal evaluation with 4 users (U1-4). We started each session with a 10-minute demonstration of our interface. Then, we gave users a short article about a technical subject: \textit{What
is a Decibel?} from \textit{howstuffworks.com} \cite{howstuffworks} or \textit{How Lasers Work} from
David Macaulay's illustrated book, \textit{The Way Things Work}
\cite{macaulay1999way}. The users' task was to create an explanatory audio recording about the subject using our interface. Users were allowed to refer to the article during the authoring process or to take notes on the master-script, but they were discouraged from recording the article by reading it out loud. We examined the users' workflow and solicited written qualitative feedback about the authoring experience at the end of the session. Each session lasted about 40 minutes.

While the size of our user evaluation is small, the initial findings (below) are extremely encouraging. All users successfully produced a complete audio recording summarizing the article. %(Table 1). 



\textbf{\systemname\ supports various workflows.} Interestingly, each user adapted a very different workflow. For example, U1 started by writing a complete list of main points. For each take, U1 recorded a few points from the list, merged them into the master-script, and then continued to record the next points on a separate take. In contrast, U2 wrote part of the script, recorded that portion, and moved on to write the script of the next part. U3 did not write an initial script, but improvised the recording and used that as a starting point to edit and re-record afterwards. Sometimes users typed verbatim script to read aloud during the recording, and other times they wrote rough outlines. For example, U2 noted,  \textit{``For
the introduction, I had a pretty good idea of what I wanted to
say, so it saved me time to use only bullet points. [For the
second part] I wrote full sentences, as I was not familiar with
all the technical details and it would have been more difficult
to improvise. I enjoyed being able to use the master-script in
both ways.''} The differences in the workflows could be due to personal preference, and/or  to the article content. In any case, our interface was able support various workflows. 

\textbf{The master-script facilitates iterative script editing and audio recording.} As the above example workflows also demonstrate, users took advantage of the master-script to go back and forth between scripting and audio recording. For instance, U2 initially wrote a very rough outline for the script.
After recording and merging the first take based on this rough script, U2 refined the master-script, and then recorded more takes. Similarly, after recording and merging audio takes into the final track, U1 noticed a mistake in the speech
(i.e., instead of saying \textit{140 decibels}, U1 had said \textit{40}
decibels). U1 corrected the corresponding recorded text in the
master-script, re-recorded
the relevant portion part by reading out the edited master-script, and replaced it.
During the back-and-forth iteration, users took advantage of our color-coded visualization that indicated sections of the mater-script that required recording  (grey) or re-recording (blue italics). 


\textbf{Users found the master-script useful, and were satisfied with the final voice-over quality.} All users offered strong positive feedback about our authoring
interface, and said they would use it to create speech recordings. They were most enthusiastic about the integration
of the script and the recordings in the master-script document,
and the ability to align the master-script to the transcripts.
To quote from one user, \textit{``Writing the script on the same interface and having
that integrated with the audio was most helpful.''}  Another user noted
that the \textit{``compare-view helped to keep track of what pieces
of information was already recorded and which ones were still
needed.''} Participants were satisfied with the overall quality
of the final recording. One user wrote, \textit{``I was surprised
how the final recording from the multiple takes was seamless.''} Users noted that while speech recognition was imperfect, \textit{``the transcriptions were accurate enough to understand and easy to check [by clicking to listen to the corresponding audio].''}

