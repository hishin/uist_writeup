\section{Results}
We have used our tool to compose several audio recordings in a variety of scenarios, including merging from pre-recorded takes, recording with a rough script, recording without a script, and when two people are collaborating on a recording. Here, we report in detail about the most interesting case, the collaborative scenario. The other cases are covered in the informal evaluation with users. 

Two of the authors (A1 and A2) of this paper collaborated on \systemname\ to create a voice-over describing \systemname. First, A1 created a rough outline on the master-script and used that to produce\ an initial recording. The result, including the master-script and the transcripts from the raw audio takes (both the texts and the corresponding audios), was passed on to A2 via a URL link. A2 first listened to the master-script audio whiling correcting several transcriptions errors in the master-script text. This did not affect the audio. Then A2 made more significant edits to the master-script. Just like editing a text document, A2 inserted new content, deleted parts, and changed the wording of sentences. The new content included verbatim speech to be recorded, as well as rough outlines (e.g. \textit{``Include examples''}). A2's edits were saved where A1 could access it also via a URL link. A1 could immediately recognize new texts that A2 inserted because it appeared in light grey. Where the edits were more subtle, A1 used the  \textit{diff-view} to compare the new master-script with the original audio takes. Finally, A1 polished the master-script again, spelling out the outlined points and re-recorded necessary parts. The new takes were merged to produce the final recording. The result is included in the supplementary material.    

As reflected in this usage case, our interface facilitates the back and forth workflow of recording and scripting by integrating the two tasks in the master-script. In our interface, recordings directly help develop the script, and vice versa. The different views (i.e. \textit{compare-view} and \textit{diff-view}) also facilitated the editing process. 
