\section{Creating Speech Recordings}

To learn about current practices and challenges for creating speech recordings, we interviewed ten professional lecturers and two video producers who regularly create audio recordings for online lectures that are published on platforms, including YouTube, Udacity, EdX and MITx. The following are several key insights we gained from these interviews.

\textbf{Scripts play a major role during recording.} All of the lecturers prepared written materials about what they were going to say before they started recording. The format and level-of-details of these scripts varied. For instance, one lecturer used his lecture slides containing images and a list of bullet points as his script. Another lecturer typed a thorough word-for-word transcription of what he was going to say in a text document. Another person used handwritten notes as an outline. In all cases, while they were recording, they kept the scripts within their view and depended on them to guide their speech.  

\textbf{Scripts evolve through the recording process.}  In many cases, the initial scripts were rough or incomplete. Only two out of the ten lecturers we interviewed prepared a word-for-word script before recording. The majority used lecture slides or handwritten notes containing a rough outline of what they were going to record. They used these outlines as guides and improvised most of the actual recorded speech. One of the lecturers did an initial recording from the outline, and then used that to flesh out the script before recording additional takes. Even when a word-for-word script was prepared beforehand, the recording often did not follow the script exactly. While recording, the speaker sometimes remembered and added more details, or found a more natural way of saying a written sentence. In some cases, major script changes were made long after the initial recording was created. For example, one lecturer noted that he periodically revisited and re-recorded parts of lectures to add up-to-date examples.

Across these scenarios, the actual recorded speech ends up differing either slightly or significantly from the initial written script.  While a few people actually edited the written script to resolve these discrepancies, in most cases the script and recorded audio end up in inconsistent states. This is partly due to the fact that editing the script and recording or editing the audio are  completely separate tasks in current workflows, which means that editing the written script after recording represents additional work.  

\textbf{The final track includes multiple recordings.} Most users recorded multiple takes and then edited them together with audio editing software to produce the final recording. Many of them noted that aligning the waveforms of the multiple takes, finding the best take of a given part, and then cutting and joining them seamlessly were very time consuming and tedious tasks.

