\section{Key Observations}

To learn about current practices and challenges for recording audio narratives, we interviewed ten professional lecturers and two video producers who regularly created audio recordings as part of online lectures that are published on platforms, including YouTube, Udacity, EdX and MITx. Following are several key insights we gained from the pilot interviews.

\textbf{Scripts play a major role during recording.} All of the lecturers  prepared written materials about what they were going to speak before they started recording. The format and level-of-details of these scripts varied. For instance, one lecturer used his lecture slides containing images and a list of bullet points as his script. Another lecturer typed a thorough word-for-word transcription of what he was going to say  in a text document.
Another person used a handwritten notes for an outline. In all cases, while they were recording, they kept the scripts within their view and depended on them to guide their speech.  

\textbf{Scripts evolve through the recording process.}
In many cases, the initial scripts were rough or incomplete. Only two out of the ten lecturers we interviewed prepared a word-for-word script before the recording. The majority of them used lecture slides or handwritten notes containing a rough outline of what they were going to record. They used these outlines as guides and improvised most of the speech. One of the lecturers did an initial recording from the outline, and then used that to flesh out the script even more before recording additional takes. Even when a thorough word-for-word script was prepared beforehand,
the recording often did not follow
the script exactly. While recording, the speaker sometimes remembered and added more details, or
found a more natural way of saying a written sentence. In some cases, a major change was made to the script after a long period of time since the recording had happened. For example, one lecturer noted that
he
periodically revisited and re-recorded parts of lectures to
include up-to-date examples. 

So, the script (i.e. what they planned to record) developed along with the audio (i.e. what they actually recorded). A few people actually edited the written script to reflect this development, while in most cases it was only reflected in the audio. This is partly due to the fact that changing the script and changing the audio is a completely separate task in current workflows. So, editing the written script is additional work.

 





\textbf{The final track is created by cutting and merging multiple recordings.} Most users recorded multiple takes, and then edited them in an audio editing software to produce the final recording. Many of them noted that aligning the waveforms of the multiple takes, finding the best take of a given part, and then cutting and joining them seamlessly  were the most time consuming and tedious tasks.


