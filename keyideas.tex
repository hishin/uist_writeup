\section{Key Ideas}

To learn about current practices and challenges for recording audio narratives, we interviewed ten professional lecturers and two video producers who regularly created audio recordings as part of online lectures that were published on platforms, including YouTube, Udacity, EdX and MITx. Following are several key insights we gained from the pilot interviews.

\textbf{Scripts play a major role during recording.} All of the lecturers  prepared written materials about what they were going to speak before they started recording. The format and level-of-details of these scripts varied. For instance, one lecturer used his lecture slides containing images and a list of bullet points as his script. Another lecturer typed a thorough word-for-word transcription of what he was going to say  in a text document.
Another person used a handwritten outline. In all cases, while they were recording, they kept the scripts within their view and depended on them to construct their narrative.  

\textbf{Script planning, recording, and editing is an iterative process.} \VTODO{Even when a thorough word-for-word script is prepared beforehand, often the recording does not follow
the script exactly. The speaker may remember and add more details while recording, or
find a more natural way of saying a written sentence. In some cases, major changes are made to the script after some recording
has happened. The iteration may happen over a longer period of time. One lecturer noted that he
periodically revisited and re-recorded parts of the lecture to include up-to-date examples.}

\textbf{The final track is created by mixing and matching multiple recordings.} Most users recorded multiple takes, and then edited them using an audio editing software to produce the final recording. Many of them noted that aligning the multiple takes, finding the best take of a given part, and seamlessly merging the cuts were the most time consuming and tedious tasks.


