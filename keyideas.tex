\section{Key Ideas}
We based our design and implementation on several key insights which we gained from a series of informal interviews with people who regularly created audio narratives. \\
\textbf{Scripts play a key role in the recording process.} All of the interviewees prepared written materials about their narrative before they started recording. The format and level-of-details of these scripts varied: it could be a list of bullet points outlining the main points to cover, or it could be a word-for-word transcription of what was going to be said. In all cases, during the recording phase, the interviewees kept their scripts within view and depended on them to construct their narrative.\\
\textbf{The final audio narrative is created by mixing and matching multiple recordings.} Most interviewees recorded multiple takes, and then edited them to produce the final recording. Many of them said that aligning the multiple takes and finding the exact place to seamlessly connect the different cuts were the most time consuming and tedious tasks.\\
\textbf{Producers iterate between the planning, recording and editing steps.} Even when a thorough word-for-word script has been prepared beforehand, the final audio narrative may not follow exactly. The speaker may add more details while recording or find a more natural way of saying the same sentence. Sometimes a major change has to be made to the script after some recording has happened. In collaborative scenarios, different persons may be working on the script and the audio iterating back and forth. One interviewee who recorded online lectures said he periodically changed and re-recorded parts of the lecture, for instance to include up-to-date examples.
\\

