\section{Conclusion and future work}
Audio recordings of speech are a prevalent form of communication
in modern media. Authoring speech
recordings involves a dynamic back and forth process between
script writing/editing and audio recording/editing. Unfortunately most existing tools treat the script and the audio as completely separate entities, which makes the back and forth workflow between them very tedious. We have present \systemname,
an authoring interface that facilitates dynamic workflows for
script writing and  audio recording/editing. \systemname\ introduces the notion of a \emph{master-script} that integrates the script with the audio and unifies the script/audio editing task. Our system supports a wide range of workflows and facilitates novice users creating speech recordings. 

Our interface focuses mainly on the content of the speech recordings and not on the details of audio quality. As one user mentioned in the feedback, we could integrate additional editing tools such as the ones in Rubin et al. \cite{rubin2013content} or \cite{rubin2015capture} to improve speech quality. 
Speech recordings are often crafted to accompany visual footage, in which case visual content is also closely related to the script/audio.
Future work could investigate how to integrate visuals in the workflow.
Since the script and audio develop over the course of many iterations, some type of version control would be useful to keep track of the development. For example, users may want to go back to a previous version before the edits, or compare different versions. 


