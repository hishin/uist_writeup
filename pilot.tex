\section{Comparative study}
We also conducted a pilot study to compare our interface with
a state-of-the-art transcript-based speech editing interface
\cite{rubin2013content}. We recruited four participants, none of whom had
experience using text-based audio editing systems. We gave them
a script with bullet points outlining a mini lecture on a science
subject (e.g. \textit{gravity} and \textit{dark matter}) and
two audio takes roughly corresponding to that script. Their task
was to cut and merge the two takes to produce a recording that
contained all the contents listed in the script and only those
contents. The two takes were similar, but both takes had some
missing content from the outline and one had some extra
content. So, the participants had to choose parts from each take
and combine them to get the final result. We encouraged the users
to focus on having the complete content rather than the details
of the audio quality (e.g. tempo, diction, flow of speech etc.).

In Rubin et al.'s system (referred to as
\textit{Interface-R} hereafter), as in our interface, users
can edit the transcript like a text document using operations
such as copy-and-paste, insert or delete, and the edits are propagated
to the audio. The system  also detects alternate takes of the
same sentence and groups them for users to select between them.
However, unlike \systemname , Interface-R is geared for editing pre-recorded
audio, and does not support scripting or real-time recording.
In fact, it only allows editing one recording at a time. To simulate multiple takes, we took advantage of their multi-column interface, so that each take appeared in a separate column one after the other as if they were spoken by two different speakers. In \systemname,  the script was contained in the master-script, whereas in Interface-R, we gave users a hardcopy of the script.

Each participant completed the task twice with different outlines,
once using our interface and the other using Interface-R.
The subject of the outline and the order of the interface was
counter-balanced. We examined the time they spent editing,
the number/type of functions they used, and the quality of the
final recording (Table~\ref{}). After the session, participants gave written
qualitative feedback about the two interfaces. In total, each
session lasted 1 hour.
   
Each of the four participants preferred \systemname\ over Interface-R for the given task, and noted they would use
our interface to edit audio recordings. Every participant also
completed the task faster using our interface (7.4 $vs$ 9.9 min Interface-R). 

\VTODO{Explain usage}

All of the participants commented on how the master-script helped them in the task. One person noted that in \systemname\ \textit{``the master-script linked the two takes into one comprehensive view
that made the editing a lot simpler, [while in Interface-R] I felt like I had to consider two separate documents and combine them manually.''} Another person preferred the \systemname\ interface for the task of editing the audio content, and commented that Interface-R would be useful to make detailed edits to the speech flow. 3 out of the 4 participants also found the color-coded visualization of the master-script (recorded vs. unrecorded) and the transcript (planned vs. improvised) very helpful. A participant wrote about the \textit{compare-view} feature that \textit{``The alignment was great. I felt like almost half of my work was
already completed before I began.''} 


