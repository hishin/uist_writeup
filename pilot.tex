\section{Comparative study}
We also conducted a pilot study to compare our interface with
a state-of-the-art transcript-based speech editing interface
\cite{rubin2013content}. \VTODO{Briefly explain Steve Rubin's
interface.} We recruited four participants, none of whom had
experience using text-based audio editing systems. We gave them
a script with bullet points outlining a mini lecture on a science
subject (e.g. \textit{gravity} and \textit{dark matter}) and
two audio takes roughly corresponding to that script. Their task
was to cut and merge the two takes to produce a recording that
contained all the contents listed in the script and only those
contents. The two takes were similar, but both takes had some
missing content from the outline and one had some extra
content. So, the participants had to choose parts from each take
and combine them to get the final result. We encouraged the users
to focus on having the complete content, rather than the details
of the audio quality (e.g. tempo, diction, flow of speech etc.).


Each participant completed the task twice with different outlines,
once using our interface and the other using Rubin et al.'s interface.
The subject of the outline and the order of the interface was
counter-balanced. We examined the time time they spent editing,
the number/type of functions they used, and the quality of the
final recording. After the session, participants gave written
qualitative feedback about the two interfaces. In total, each
session lasted one hour.
   
Each of the four participants preferred \systemname\ over Rubin
et al.'s interface for the given task, and noted they would use
our interface to edit audio recordings. Every participant also
completed the task faster using our interface (avg. 7.4 $\pm$
1.6 min) than Rubin et al's interface(9.9 $\pm$ 1.5 min). Table~\ref{evalsumary2}
summarizes the participants' usage of \systemname\ during the
editing session.

\VTODO{Explain usage}

\VTODO{Positive Feedback}
